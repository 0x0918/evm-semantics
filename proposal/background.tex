\section{Background}
 Many classes of bugs exist in Ethereum smart contracts, from transaction-ordering dependence to mishandled exceptions and even a contract running out of gas before full completion of a function. In light of this, there is a newfound interest in recent academic work to verify the correctness of smart contracts in Ethereum. Oyente~\cite{luumaking} is an example of very recent work in this area. While Oyente focused on building a symbolic execution tool from the ground up, there has not yet been work done on modeling the full semantics of EVM bytecode using modern program analsysis hammers, and performing a host of program verification techniques to verify properties of smart contracts as they execute. We are implementing our model in the K framework, a semantic framework built by researchers at the University of Illinois, Urbana-Champaign.   
 
 K is an extensible semantic framework that is used to model programming languages, type systems, and formal analysis tools~\cite{rosu-serbanuta-2010-jlap}. There a number of semantic frameworks to choose from - we are choosing to use K in our formal analysis primarily because of its sipmlicity and extensibility. Adding additional pieces of state is typically as easy as adding in new entires into a configuration and updating the rules to match. In addition, we are working closely with the developers of K (Manasvi and his lab mantain K) so to get more nuanced feedback about the rigor and correctness of our semantic model. 
% Probably recycle a lot of Oyente paper background
\todo{Why are we using K?}

