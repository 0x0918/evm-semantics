\section{Background} The growing popularity of smart contracts has led to
increased scrutiny of the security of smart contracts. Because contracts
inherently involve money, bugs in smart contract programs can be devastating to
parties involved. A example of such a catastrophe is the recent DAO
attack~\cite{dao-attack}, where \$150 million of ether was stolen, prompting an
unprecedented hard fork of the Ethereum blockchain. Many classes of bugs exist
in Ethereum smart contracts, from transaction-ordering dependence to mishandled
exceptions and even a contract running out of gas before full completion of a
function. 

Smart contracts are an attractive method for formal program verification, which
theoretically can provide guarantees on smart contract execution safety,
liveness, and other forms of application logic. In particular, K is an
extensible semantic framework that is used to model programming languages, type
systems, and formal analysis tools~\cite{rosu-serbanuta-2010-jlap}. There a
number of semantic frameworks to choose from - we are choosing to use K in our
formal analysis primarily because of its simplicity and extensibility. For
instance, adding additional pieces of state is typically as easy as adding in
new entries into a configuration and updating the rules to match. We are also
working closely with the developers of K (Manasvi and his lab mantain K) to get
more nuanced feedback about the rigor and correctness of our semantic model. 
