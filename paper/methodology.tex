\section{Methodology}

We used K and modeled the EVM, to say interesting things about EVM programs.

\subsection{Language Independent Program Verification}

Developing new tools for each new language is labor intensive and bug-prone. To
avoid some of the start-up cost of formal verification in a new language, \K
defines its tools parametrically over the input language~\cite{inproceedings}.
From the syntax and operational semantics of a language, parsers, interpreters,
debuggers, and verifiers are derived. This infrastructure is developed and
debugged once, and then instantiated to specific languages. The instantiation
requires describing for \K the way that execution state looks in your language
(the \textit{configuration}), then describing how simplification in your
language happens over that state (as transitition \textit{rules}).

\subsection{EVM Execution State}

The state of EVM is broken into two components: the local state of an active
transaction (smart contract execution), and the state of the network as a whole
(account information). To mirror this, our configuration consists of two
subconfigurations.

\subsubsection{Transaction Execution}

Each transation has an associated account, a program counter, a word-stack, and
a volatile local memory.

\begin{verbatim}

<k> ... </k>
<id> ... </id>
<pc> ... </pc>
<wordStack> ... </wordStack>
<localMem> ... </localMem>
<callStack> ... </callStack>

\end{verbatim}

The \texttt{<k>} cell holds the next execution step of the EVM.  The
\texttt{<id>} cell holds the ID of the account the EVM contract belongs to and
the \texttt{<pc>} cell holds the current program counter. The
\texttt{<wordStack>} is the simple stack available to the EVM program being
executed, and the \texttt{<localMem>} cell holds the volatile auxillary memory
available to the program. The \texttt{<callStack>} cell records the calls made
across a transaction.

\subsubsection{Network State}

The network is modelled as a map from account ids to their associated program,
cold storage, and current balance. This can be thought of as the blockchain
state at any given time.

\begin{verbatim}

<accounts>
    <account multiplicity="*">
        <acctID> .AcctID </acctID>
        <program> .Map </program>
        <storage> .Map </storage>
        <balance> 0:Word </balance>
    </account>
</accounts>

\end{verbatim}

The \texttt{<accounts>} cell holds information about the accounts on the block
chain. Each \texttt{<account>} holds the program associated with it in the
\texttt{<program>} cell, its permanent storage in the \texttt{<storage>} cell,
and the balance associated with the account in the \texttt{<balance>} cell.

\subsection{Implementing Instructions}
After designing our K configuration, we then implemented a large subset of the
EVM instructions detailed in the Ethereum Yellow Paper~\cite{gavwood}. At the
point of writing, we have implemented almost 60\% of the instructions in EVM,
and can run a vast number of EVM programs using our subset of instructions. In
this section, we discuss the classes of instructions we have already implemented, as
well as outstanding instructions that need to be implemented.

Throughout the model, we implement several classes of instructions. The first
kind of instruction deals with operations on the stack. We group instructions
based on how many elements they pull off of the stack, for example, a
\textit{BinaryStackOperation} pulls two elements off of the stack, performs some
computation on those elements, and then pushes the elements back onto the stack.

Other instructions may use the word stack but also make use of other parts of
the K configuration, for example, the programCounter or localMemory. A JUMP
instruction touches the programCounter to jump to previous instructions; a MLOAD
instruction will load a word from a program\textquotesingle s local memory and move it to the
stack.

\todo{Something about function calls}

\todo{Everett, Manasvi}
