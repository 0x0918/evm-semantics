\section{Methodology}
We used K and modeled the EVM, to say interesting things about EVM programs.
\subsection{Language Independent Program Verification}
\todo{Everett}
\subsection{Our configuration and design}
\subsubsection{Configuration}
We closely model the configuration described in the ethereum yellow paper. Our configuration can mainly be thought of consisting of two subconfigurations. The first subconfiguration models a transaction in the network.
\begin{verbatim}

<k> $PGM:EVMSimulation </k>
<accountID> .AcctID </accountID>
<pc> 0 </pc>
<wordStack> .WordStack </wordStack>
<localMem> .Map </localMem>
<callStack> .CallStack </callStack>

\end{verbatim}

The <k> cell holds the instruction of the current EVM program being executed. The <accountID> cell holds the ID of the account the EVM contract belongs to, and the <wordStack> is the simple stack available to the EVM program being executed. The <localMem> cell holds the volatile auxillary memory available to the program, and the <callStack> cell holds the methods calls made across a transaction.

\begin{verbatim}

<accounts>
    <account multiplicity="*">
        <acctID> .AcctID </acctID>
        <program> .Map </program>
        <storage> .Map </storage>
        <balance> 0:Word </balance>
    </account>
</accounts>

\end{verbatim}

The second subconfiguration models the state of the blockchain at any given time. The <accounts> cell holds information about the accounts on the block chain. Each <account> holds the program associated with it in the <program> cell, its permanent storage in the <storage> cell and the balance associated with the account in the <balance> cell.

\subsection{Implementing Instructions}
After designing our K configuration, we then implemented a large subset of the
EVM instructions detailed in the Ethereum Yellow Paper~\cite{gavwood}. At the
point of writing, we have implemented almost 60\% of the instructions in EVM,
and can run a vast number of EVM programs using our subset of instructions. In
this section, we discuss the classes of instructions we have already implemented, as
well as outstanding instructions that need to be implemented.

Throughout the model, we implement several classes of instructions. The first
kind of instruction deals with operations on the stack. We group instructions
based on how many elements they pull off of the stack, for example, a
\textit{BinaryStackOperation} pulls two elements off of the stack, performs some
computation on those elements, and then pushes the elements back onto the stack.

Other instructions may use the word stack but also make use of other parts of
the K configuration, for example, the programCounter or localMemory. A JUMP
instruction touches the programCounter to jump to previous instructions; a MLOAD
instruction will load a word from a program\textquotesingle s local memory and move it to the
stack.

\todo{Something about function calls}

\todo{Everett, Manasvi}
