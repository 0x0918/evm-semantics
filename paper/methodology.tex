\section{Methodology}

Instantiation of \K{} to EVM requires modeling transaction execution state and
network evolution using the \textit{configuration}, and detailing
changes in transaction execution and network evolution using transition
\textit{rules}. In this section, we outline the design of our \K{} model and the
design decisions implemented for our tool.

\subsection{EVM Execution State}

The state of EVM is broken into two components: the state of an active
transaction (smart contract execution), and the state of the network as a whole
(account information). To mirror this, our configuration consists of two pieces.

\subsubsection{Transaction Execution}

Each transaction has an associated account, a program counter, a word-stack, and
a volatile local memory.

\begin{verbatim}
<k> ... </k>
<id> ... </id>
<pc> ... </pc>
<gas> ... </gas>
<wordStack> ... </wordStack>
<localMem> ... </localMem>
<callStack> ... </callStack>
\end{verbatim}

The \texttt{<k>} cell holds the next execution step of the EVM. The
\texttt{<id>} cell holds the ID of the account the EVM contract belongs to and
the \texttt{<pc>} cell holds the current program counter. The \texttt{<gas>}
cell holds the current gas consumption of the executing contract. The
\texttt{<wordStack>} is the simple stack available to the EVM program being
executed, and the \texttt{<localMem>} cell holds the volatile auxiliary memory
available to the program. The \texttt{<callStack>} cell records the calls made
across a transaction.

\subsubsection{Network State}

The network is modeled as a map from account ids to their associated program,
cold storage, and current balance. This can be thought of as the blockchain
state at any given time.

\begin{verbatim}
<accounts>
    <account multiplicity="*">
        <acctID> ... </acctID>
        <program> ... </program>
        <storage> ... </storage>
        <balance> ... </balance>
    </account>
</accounts>
\end{verbatim}

The \texttt{<accounts>} cell holds information about the accounts on the blockchain. Each \texttt{<account>} holds the program associated with it in the
\texttt{<program>} cell, its permanent storage in the \texttt{<storage>} cell,
and the balance associated with the account in the \texttt{<balance>} cell.

\subsection{EVM Transition Rules}

EVM smart contracts consist of a sequence of commands for effecting the local
execution state and the global network state. Commands are grouped using \K{}'s
sorts; for example we have the \texttt{StackOp} sort for stack operators, the
\texttt{ControlFlowOp} sort for operations which affect control flow (e.g.
calling other contracts), and the \texttt{LocalOp} sort for other operations
that may effect the local transaction state. At the point of writing, we have
implemented almost 60\% of the EVM opcodes.

Semantics of EVM instructions are implemented using the Ethereum Yellow Paper as a
reference~\cite{wood2014ethereum}. They are provided as \K{} rules over the
execution state. For example, below are the rules for \texttt{MLOAD} and
\texttt{MSTORE}:

\begin{verbatim}
rule <k> MSTORE => #updateLocalMem INDEX VALUE ... </k>
     <wordStack> INDEX : VALUE : WS => WS </wordStack>

rule <k> MLOAD => . ... </k>
     <wordStack> INDEX : WS => VALUE : WS </wordStack>
     <localMem>... INDEX |-> VALUE ...</localMem>
\end{verbatim}

The first rule states that if the next instruction is \texttt{MSTORE}, and the
top two elements of the word stack are \texttt{INDEX} and \texttt{VALUE}, then
they should be removed from the stack and the next instruction should be
\texttt{\#updateLocalMem INDEX VALUE} (an auxiliary internal command for working
with the local memory).

The second rule states that if the next instruction is \texttt{MLOAD}, and the
top element of the stack is \texttt{INDEX}, and somewhere in the local memory we
have the binding \texttt{INDEX |-> VALUE}, then \texttt{INDEX} should be
replaced with \texttt{VALUE} on the stack. Notice that we take advantage of \K{}'s
\textit{configuration abstraction}, which allows each rule to only specify the
parts of the state relevant to it.

\subsubsection{Inter-contract Execution}

EVM allows contracts to call each other, accounting for the rich dynamics
possible in the Ethereum network but also providing a wealth of potential bugs
and pitfalls in contract design. The next two rules model one contract making a
\texttt{CALL} to another. Similar rules for \texttt{RETURN}, which ends
execution of the current function call and passes some values back to the
caller, are defined in our semantics.

\begin{verbatim}
rule <k> CALL => #processCall
        { ACCT | ETHER | #range(LM, INIT, SIZE) } ...
     </k>
     <wordStack> ACCT : ETHER : INIT : SIZE : WS => WS </wordStack>
     <localMem> LM </localMem>
\end{verbatim}

First, we gather the arguments to \texttt{CALL}, which consists of which process
to call (\texttt{ACCT}), the amount of Ether to transfer (\texttt{ETHER}), and
the region of local memory to use as arguments (\texttt{\#range(LM, INIT,
SIZE)}). \texttt{\#processCall} is used as an internal command which has the
following effects:

\begin{verbatim}
rule <k> #processCall {ACCT | ETHER | WL}
      =>    #decreaseAcctBalance CURRACCT ETHER
         ~> #increaseAcctBalance ACCT ETHER
         ~> #pushCallStack
         ~> #setProcess {ACCT | 0 | .WordStack | #asMap(WL)}
     ... </k>
     <id> CURRACCT </id>
\end{verbatim}

The current accounts balance is reduced (\texttt{\#decreaseAcctBalance CURRACCT
ETHER}), then the callee's account balance is increased, the current execution
state is pushed onto the \texttt{<callStack>} using internal command
\texttt{\#pushCallStack}, and the callee is set as the current process. The
operator \texttt{\textasciitilde >} can be read as "followed by", similar to the semicolon in many programming languages.
