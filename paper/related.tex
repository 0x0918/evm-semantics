\section{Related Work}
Formal verification of programming languages is not a particularly new idea,
and our design decisions presented in this work are certainly influenced by
prior work in this area. For example, the semantics of many other programming
languages have been implemented in \K{}, such as
C~\cite{ellison-2012-thesis}, Java~\cite{bogdanas-rosu-2015-popl},
and JavaScript~\cite{park-stefanescu-rosu-2015-pldi}. Formal verification
techniques using \K{} are also able to successfully verify properties about these
programming languages. For example, Hathhorn et.
al~\cite{hathhorn-ellison-rosu-2015-pldi} demonstrate using formal verification
techniques to successfully identify undefined C compiler behavior and alert
program writers if they are inadvertently introducing undefined behavior in
their C programs. Similarly, Stefanescu et al.~\cite{stefanescu-park-yuwen-li-rosu-2016-oopsla} presented a language independent verification framework for program verification, such that the language semantics alone could be used for program verification.

There has been recent effort and interest in formally verifying properties of
smart contracts. Luu et~al.~\cite{luumaking} formalized a subset of the EVM,
called EtherLite, and built a symbolic execution tool to check for common bugs
in smart contracts. Their semantics, however, do not consider the role of gas in
Ethereum transactions, and cannot be used to prove properties concerning gas on
smart contracts. Bhargavan et~al.\cite{evmf*} propose a method for formally
verifying EVM smart contracts by converting Solidity programs or EVM bytecode to
F*, a functional programming language focused on program verification.
