\section{Background}

\subsection{Ethereum Virtual Machine (EVM)}

\todo{Zzzzma}

\subsection{The \K{} Framework}

Developing tools for each new language is labor intensive and error-prone. \K{}
defines its tools parametrically over the input language, avoiding some of this
start-up cost. Parsers, interpreters, debuggers, and verifiers are derived
directly from the syntax and operational semantics of the language. This
infrastructure is developed once, then instantiated to specific
languages.~\cite{stefanescu-park-yuwen-li-rosu-2016-oopsla}

Developing a new language in \K{} requires the definition of that language's
syntax (given as a BNF-style grammar), as well as the operational semantics of
that language (as transition rules). \K{} provides several facilities for making
this definition easier, including \textit{configuration abstraction};
configuration abstraction allows each transition rule to only mention the parts
of execution state needed for that transition.

Once a language is defined, \K{} can read and execute programs in that language
both on concrete and symbolic inputs, giving an interpreter and a symbolic
execution engine, respectively. These can be regarded as a reference
implementation of the language. \K's Reachability Logic prover can also be used
to verify functional correctness and safety properties of
programs.~\cite{stefanescu-ciobaca-mereuta-moore-serbanuta-rosu-2014-rta}
Reachability queries are provided to the prover using the same syntax as a \K{}
rule, and reduced to a query to the SMT solver Z3.~\cite{de2008z3}

\K{} semantics exist for several large languages, most notably C, Java, and
JavaScript, as well as for many smaller languages. These semantics are
continuously refined as bugs are found, either in the original language
specification or in the \K{} implementation. By providing an executable and
testable specification of a language semantics, language design issues can be
uncovered and fixed early. Additionally, tools for checking for common bugs and
program errors can be developed to steer developers away from these pitfalls.
For example, C has a large amount of undefined and implementation defined
behavior, which is documented in a \K{} definition so that developers can be
informed when their program may include such behavior
\cite{guth-hathhorn-saxena-rosu-2016-cav}.

The semantics of EVM in \K{} lay the groundwork for rigorous and practical
debugging and analysis of both EVM smart contracts and the EVM network. Tools
for checking EVM contracts for common bugs can be developed in a high-level yet
semantically-rigorous way. This is valuable to users of the EVM network, where
errors in contracts directly means that value can be lost.

\todo{Everett + Manasvi}
