\section{Background}

\subsection{Ethereum Virtual Machine (EVM)}

Ethereum is a public, distributed ledger based on blockchain technology first
proposed and popularized by Bitcoin. Whereas Bitcoin's blockchain only stores
transactions that exchanges Bitcoin between addresses, Ethereum's blockchain
stores the complete execution state of distributed programs known as Ethereum
Virtual Machine (EVM) \textit{smart contracts}. Smart contracts, which are often
written in higher level languages such as Solidity~\cite{solidity} or
Serpent~\cite{serpent}, are then compiled to EVM
bytecode, a Turing-complete set of 65 unique opcodes~\cite{wood2014ethereum}. Smart
contracts consist of a \textit{contract address}, contract \textit{balance}, and
private program execution code and state.

Beyond their ability to execute arbitrary logic, smart contracts have two
distinct features that enable their practical usage. First, smart contracts
include operations that allow participants to transfer \textit{Ether}, the
currency of Ethereum, between smart contracts. These also include individual user
accounts because they are essentially a limited type of smart contract. Second, to
incentivize Ethereum blockchain miners to execute EVM smart contracts, every
opcode instruction requires a certain amount of \textit{gas}, a special form of
Ether, to be paid to the miner. Gas requirements for program execution also
provide the side benefit of deterring resource-exhaustion attacks. These two
features provide both an incentive and a mechanism for creating smart contracts
that handle substantial amounts of Ether; certain contracts have
amassed over \$100~million.

\subsection{The \K{} Framework}

Developing tools for each new language is labor intensive and error-prone. \K{}
defines its tools parametrically over the input language, avoiding some of this
start-up cost. Parsers, interpreters, debuggers, and verifiers are derived
directly from the syntax and operational semantics of the language. This
infrastructure is developed once, then instantiated to specific
languages.~\cite{stefanescu-park-yuwen-li-rosu-2016-oopsla}

Developing a new language in \K{} requires the definition of that language's
syntax (given as a BNF-style grammar), as well as the operational semantics of
that language (as transition rules). \K{} provides several facilities for making
this definition easier, including \textit{configuration abstraction};
configuration abstraction allows each transition rule to only mention the parts
of execution state needed for that transition.

Once a language is defined, \K{} can read and execute programs in that language
both on concrete and symbolic inputs, producing an interpreter and a symbolic
execution engine, respectively. These can be regarded as a reference
implementation of the language. \K{}'s Reachability Logic prover can also be used
to verify functional correctness and safety properties of
programs.~\cite{stefanescu-ciobaca-mereuta-moore-serbanuta-rosu-2014-rta}
Reachability queries are provided to the prover using the same syntax as a \K{}
rule, and reduced to a query to the SMT solver Z3.~\cite{de2008z3}

%Added this paragraph to related works.
%\K{} semantics exist for several large languages, most notably C, Java, and
%JavaScript, as well as for many smaller languages. These semantics are
%continuously refined as bugs are found, either in the original language
%specification or in the \K{} implementation. By providing an executable and
%testable specification of a language semantics, language design issues can be
%uncovered and fixed early. Additionally, tools for checking for common bugs and
%program errors can be developed to steer developers away from these pitfalls.
%For example, C has a large amount of undefined and implementation defined
%behavior, which is documented in a \K{} definition so that developers can be
%informed when their program may include such behavior
%\cite{guth-hathhorn-saxena-rosu-2016-cav}.

The semantics of EVM in \K{} lay the groundwork for rigorous and practical
debugging and analysis of both EVM smart contracts and the EVM network. Tools
for checking EVM contracts for common bugs can be developed in a high-level yet
semantically-rigorous way. This is valuable to users of the EVM network, where
errors in contracts often directly mean fiscal loss.
