\section{Introduction}
The growing popularity of smart contracts has led to an increased scrutiny of
the security of smart contracts. Contracts inherently involve money, so bugs in
such programs can often be devastating to the involved parties. An example of
such a catastrophe is the recent DAO attack~\cite{dao-attack}, where \$150
million worth of ether was stolen, prompting and unprecedented hard fork of the
ethereum blockhain. While devastating, the bug that caused such an attack is
well known as a common bug in ethereum smart contracts. In fact, many such
classes of bugs exist in smart contracts, ranging from transaction-ordering
dependence to mishandled exceptions. Even a contract running out of gas before
the full completion of a function can interfere with the correctness of smart
contracts, and so it is important to do so in order to protect the integrity of
ethereum to its many users.

As the smart contract ecosystem grows in complexity, there is an additional
reliance on code from external sources in order to enhance the development cycle
for smart contracts. While this code is ostensibly correct, there is in fact no
mechanism for verifying the correctness of these programs, as implementation
details can differ drastically from source to source. from For example,
developers often include code from external libraries or even from sources such as
StackOverflow to include in their ethereum contracts.

Smart contracts are an attractive method for formal program verification, which
can theoretically provide guarantees on smart contract execution safety,
liveness, and other forms of application logic. In particular, K is an
extensible semantic framework that is used to model programming languages, type
systems, and formal analysis tools~\cite{rosu-serbanuta-2010-jlap}. While there
are a number of semantic frameworks to choose from, we are choosing to use K in
our formal analysis primarily because of its simplicity and extensibility.

In this paper, we build a K~model of the Ethereum Virtual Machine (EVM) for use
in verifying properties on smart contracts. The need for verification of these
programs is well motivated, and to our knowledge, no such verification has been
done using the power of modern verification tools in the academic literature.
The contributions of this paper are as follows:

\begin{itemize}
\item{\bf EVM Semantics in K} We implement EVM semantics in the K~framework, and
cover a large subset of EVM instructions and their properties.
\item{\bf EVM Program Verification} We create a simple EVM program that sums
values from 1 to some input N, and verify that the program for correctness. We then show how this can be extended to any given EVM program.
\item{\bf EVM Property Checking} We show that certain properties, for example
minimum gas required for proper execution, can be posited by way of our system.
\end{itemize}
