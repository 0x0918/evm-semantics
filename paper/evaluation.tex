\section{Evaluation}
In order to explore the generated deductive program verifier, we wrote an EVM
program that finds the sum of a given number $n \geq 0 $, and that when called, the
value the program returned was $ \sum_{i = 0}^{n} i =  \frac{n \times  (n + 1 )}{2}$. The deductive
prover in \K{} requires a specification file, written over as a reachability
claim. Our specification file consists of two claims - the first rule claims
that from a state when we call the EVM program to be verified, we end up in a state in which the program return the correct value.  

\begin{verbatim}
rule
<T> 
    <k> CALL => STOP </k> 
    <id> 1 </id> 
    <pc> 10 => 11 </pc> 
    <gas> _ => _ </gas>
    <gasPrice> 1 </gasPrice> 
    <callStack> .CallStack </callStack> 
    <wordStack> 2 : ( 4000 
            : ( 0 : ( 1 : ( 1 : ( 1 : .WordStack ) ) ) ) ) 
               => _ 
    </wordStack> 
    <localMem> (0 |-> X) 
               => (0 |-> X 1 |-> (X *Int (X +Int 1)) /Int 2) 
    </localMem>
    <accounts>
        <account>     
            <acctID> 1 </acctID> 
            <program> Account1Pgm </program>
            <storage> .Map </storage>
            <balance> 40 </balance> 
        </account>
        <account>
            <acctID> 2 </acctID> 
            <program> Account2Pgm </program>
            <storage> .Map </storage>
            <balance> 40 </balance> 
        </account>
    </accounts>
</T>
\end{verbatim}

The second reachability rule establishes the circularity, and claims that for
every loop iteration, the value of the sum in location $1$ is in fact a partial
sum of the number of iterations at that step in the program. The second
reachability claim is the equivalent of a loop invariant in reachability logic.

\begin{verbatim}
rule
<T> 
    <k> JUMP1 => STOP </k> 
    <id> 2 => 1 </id> 
    <pc> 7 => 11 </pc> 
    <gas> _ => _ </gas>
    <gasPrice> 1 </gasPrice> 
    <callStack>  { 1 | 10 | _ | 1 : 
             ( 1 : .WordStack ) | 0 |-> X } .CallStack
        => .CallStack  
    </callStack> 
    <wordStack> 10 : ( A : .WordStack ) => .WordStack </wordStack>
    <localMem> 0 |-> A 1 |-> B 
        => (0 |-> X 1 |-> (B +Int (A *Int (A +Int 1)) /Int 2)) 
    </localMem>
    <accounts>
        <account>     
            <acctID> 1 </acctID> 
            <program>
                Account1Pgm 
            </program>
                <storage> .Map </storage>
            <balance> 40 </balance> 
        </account>
        <account>
            <acctID> 2 </acctID> 
            <program> 
                Account2Pgm
            </program>  
            <storage> .Map </storage>
            <balance> 40 </balance> 
        </account>
    </accounts>
</T>

\end{verbatim}
\begin{verbatim}

Account1Pgm

0 |-> ( PUSH X ) 1 |-> ( PUSH 0 ) 2 |-> MSTORE 
3 |-> ( PUSH 1 ) 4 |-> ( PUSH 1 ) 5 |-> ( PUSH 1 ) 
6 |-> ( PUSH 0 ) 7 |-> ( PUSH 4000 ) 8 |-> ( PUSH 2 ) 
9 |-> CALL 10 |-> STOP 

            
Account2Pgm

0 |-> ( PUSH 0 ) 1 |-> ( PUSH 1 ) 2 |-> MSTORE 
3 |-> ( PUSH 0 ) 4 |-> MLOAD 5 |-> ( PUSH 10 ) 
6 |-> JUMP1 7 |-> ( PUSH 1 ) 8 |-> ( PUSH 1 ) 
9 |-> RETURN 10 |-> ( PUSH 1 ) 11 |-> MLOAD 
12 |-> ( PUSH 0 ) 13 |-> MLOAD 14 |-> ADD 
15 |-> ( PUSH 1 ) 16 |-> MSTORE 17 |-> ( PUSH 1 ) 
18 |-> ( PUSH 0 ) 19 |-> MLOAD 20 |-> SUB 21 |-> ( PUSH 0 ) 
22 |-> MSTORE 23 |-> ( PUSH 3 ) 24 |-> JUMP
\end{verbatim}
